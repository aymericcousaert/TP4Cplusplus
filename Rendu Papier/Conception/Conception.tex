\documentclass{article}



\usepackage[latin1]{inputenc}    
\usepackage[T1]{fontenc}
\usepackage[francais]{babel}     
\usepackage{listings}
\usepackage[margin=2.5cm]{geometry}

\date{}
\title{Analyse de logs apache - Document de Conception}
\author{Aymeric Cousaert et Mael Risbourg}

\begin{document}
\maketitle

\vspace{1cm}
\tableofcontents
\vspace{1cm}

\begin{section}{Spécifications de l'application}
Afin de réaliser notre projet, nous avons tout d'abord dû comprendre quel était le but du programme et quels différents cas d'execution pouvons nous rencontrer. Nous avons donc spécifié ces cas  que nous pouvons classé en trois catégories :

Premièrement, il y a les cas normaux, ceux pour lesquelles notre programme est censé fonctionner sans aucun problème et qui avec un bon manuel d'utilisation devrait arriver le plus fréquemment. Ici, nous considérons que la syntaxte des commandes est toujours correctement respectée. Il y a donc le cas où seul le nom du dossier texte (.log ou .txt) est donné et notre programme affiche les 10 sites les plus consultés. Il y a ensuite les cas où une option est rajoutée en paramètre. Notre programme affichera toujours le top 10 mais créera en plus le fichier .dot si l'option -g est indiqué; filtrera les sites n'étant pas des images si -e est indiqué; ou encore filtrera seulement les consultations de sites à une certaines heures si -t est indiqué avec l'heure voulu. Enfin, plusieurs options peuvent être indiqué en même temps dans tel cas notre programme sera capable de toutes les traiter correctement.

Ensuite, il y a les cas limites qui vont en réalité dépendre des données recu dans le fichier texte. Nous avons relevé 3 cas limites. Tout d'abord, si l'execution avec les options ne débouche sur aucun résultats, nous l'indiquons à l'utilisateur et il n'y a donc aucun top 10 possible a afficher. Egalement, si l'execution donne quelques resultats mais pas suffisament pour afficher les 10 sites les plus consultés, nous indiquerons à l'utilisateur le classement avec seulement les sites possibles et donc celui-ci aura une taille inférieur à 10. Par ailleurs nous devons aussi prendre en compte le cas des exeaquos. Ainsi, si deux sites ont le même nombre de vues, nous choisissons de leur donner deux classement différents mais comme notre classement affichera aussi le nombre de visites, l'utilisateur sera capable de voir que les deux sites sont exaequos et il n'y aura donc pas d'ambiguïté. De plus, si l'égalité concerne le 10ème et le 11ème site, notre programme affichera un seul de ces sites afin de garder un top 10 et ne pas afficher trop de sites si il y a de nombreux exaequo.


\begin{section}{Architecture globale de l'application}
Nous avons utilisés 3 classes : Ligne, Classement, et Graphe. La classe Ligne possède tous les attributs d'une ligne d'un journal de logs. La classe Classement possède un seul attribut qui est un tableau des 10 cibles les plus demandées. La classe Graphe possède également un seul attribut qui est une map dont le premier élément est un document accessible et le second et le nom donné au noeud qui lui est associé dans le graphe.
\end{section}


\begin{section}{Structures de données commentées et justifiées}

\end{section}
\end{document}
